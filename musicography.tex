\documentclass{article}

\usepackage[T1]{fontenc}
\usepackage{lmodern}
\usepackage{geometry}
\usepackage{semantic-markup}
\usepackage[bigger]{musicography}
\usepackage{fancyvrb}
\frenchspacing
\setcounter{secnumdepth}{1}

\NewDocumentEnvironment{codetable}{}
{
    \begin{quote}
        \begin{tabular}{p{0.3\textwidth} p{0.4\textwidth} l}
}
{
        \end{tabular}
    \end{quote}
}

\usepackage[
    pdftitle={The musicography Package: Symbols for Music Writing},
    pdfauthor={Andrew A. Cashner},
    pdfsubject={LaTeX package}, pdfkeywords={LaTeX, symbols, music, musicology, humanities}
]{hyperref}

\title{The \code{musicography} Package: Symbols for Music Writing with
\code{pdflatex}}
\author{Andrew A. Cashner\thanks{%
    \href{mailto:andrewacashner@gmail.com}
    {\nolinkurl{andrewacashner@gmail.com}}%
    }%
}
\date{September 8, 2023}

\begin{document}
\maketitle

Most \LaTeX{} font packages and freely available fonts only provided a limited
range of musical symbols. 
The \code{lilyglyphs} package uses Lilypond's fonts, but requires
\code{lualatex}. 
This package makes available the most commonly used symbols in writing about
music in a way that looks consistent and attractive.
It is especially useful for \code{pdflatex} but also works alongside Unicode
fonts with \code{xelatex}.
It includes accidentals, meters, notes of different rhythmic values, and
automatic formatting of figured bass.

This package builds on the approach used in the \code{harmony} package, where
the symbols are taken from the MusiXTeX fonts.
But it provides a larger range of symbols and a more flexible, user-friendly
interface written using the new interface introduced in  \code{xparse} (now
built in to \LaTeX).
The package is meant to work hand-in-hand with my \code{semantic-markup}
package, and overrides some of its music-symbol commands with better
alternatives.

\tableofcontents

\section{Package Options}

To use the package, write \verb|\usepackage{musicography}| in your preamble.
If you are also using this author's \code{semantic-markup} package, 
\code{musicography} will automatically redefine the commands for accidentals in
the other package.

The \code{bigger} option provides larger font sizes that match better with
certain fonts (tested with 12-point \code{ebgaramond}).

\section{Fonts}

The package requires the MusiX\TeX{} fonts, which are installed by default in
\TeX{}Live.
In case there is anyone else using the obscure \verb|\meterCZ| command (in
which case, send me a note!) it now requires the EB Garamond font, also
installed in \TeX{}Live.

Most of the font commands can be customized by redefining one of the following
commands:
\begin{itemize}
    \item \verb|\musAccidentalFont|: The size of accidentals is defined as
        larger relative to the other symbols.

    \item \verb|\musNumFont|: This command is used in typesetting time
        signatures and meters. By default such numbers are boldfaced, which
        looks better in most fonts. Redefine this command to adjust the
        appearance of the numbers, for example by using lining numerals.
        Recommended with \code{ebgaramond} or \code{libertine}:
        \begin{quote}
            \verb|\RenewDocumentCommand{\musNumFont}{\liningnums}|
        \end{quote}
        The command is employed in a way that both font switches and commands
        with arguments will work (e.g., \verb|\bfseries\liningnums|), but take
        care when combining multiple elements.

    \item \verb|\musFigSize| and \verb|\musFigFont|: These commands are used
        in typesetting figured bass. By default \verb|\musFigSize| sets a
        smaller size for both numerals and accidentals. The command
        \verb|\musFigFont| is empty by default but it would be a good idea to
        make sure the treatment of numerals matches \verb|\musNumFont| if, for
        example, you are using lining numerals.
\end{itemize}


\section{Symbols and Commands}

\subsection{Accidentals}

\begin{codetable}
    Flat & \verb|\musFlat| or \verb|\fl| & \musFlat\\
    Sharp & \verb|\musSharp| or \verb|\sh| & \musSharp\\
    Natural & \verb|\musNatural| or \verb|\na| & \musNatural\\
    Double Flat & \verb|\musDoubleFlat| & \musDoubleFlat\\
    Double Sharp & \verb|\musDoubleSharp| & \musDoubleSharp\\
\end{codetable}

\subsection{Notes of Different Rhythmic Values}

Commands are available using modern (United States) note names; in several
cases there are also aliases for older note names.

\begin{codetable}
    Whole note (semibreve) & \verb|\musWhole| or \verb|\musSemibreve| &
    \musWhole\\
    Half note (minim) & \verb|\musHalf| or \verb|\musMinim| & \musHalf\\
    Quarter note (semiminim) & \verb|\musQuarter| or \verb|\musSeminimin| &
    \musQuarter\\
    Eighth note (corchea) & \verb|\musEighth| or \verb|\musCorchea| &
    \musEighth\\
    Sixteenth note & \verb|\musSixteenth| & \musSixteenth\\
    Thirty-second note & \verb|\musThirtySecond| & \musThirtySecond\\
    Sixty-fourth note & \verb|\musSixtyFourth| & \musSixtyFourth\\
\end{codetable}

A dot may be added to any of the above by adding \code{Dotted} to the end of the
command. For example:

\begin{codetable}
    Dotted whole note & \verb|\musWholeDotted| & \musWholeDotted\\
    Dotted quarter note & \verb|\musQuarterDotted| & \musQuarterDotted\\
    Dotted thirty-second note & \verb|\musThirtySecondDotted| &
    \musThirtySecondDotted\\
\end{codetable}

\subsection{Other Symbols}

\begin{codetable}
    \emph{Segno} repeat marker & \verb|\musSegno| & \musSegno \\
\end{codetable}

\subsection{Meter Signatures}

\begin{codetable}
    Common duple & \verb|\meterC| & \meterC\\
    \term{Alla breve} & \verb|\meterCutC| & \meterCutC\\
    Ternary (16th--18th cent.) & \verb|\meterCThree| & \meterCThree\\
    Ternary with $3:2$ proportion & \verb|\meterCThreeTwo| &
    \meterCThreeTwo\\
    Spanish 17th-cent. ternary & \verb|\meterCZ| & \meterCZ\\
    Medieval \term{tempus perfectum} & \verb|\meterO| & \meterO\\
\end{codetable}

For other time signatures, use \verb|\musMeter{}{}|; the two arguments are the
numerator and denominator of the fraction, respectively.

By default this command applies the \verb|\musNumFont| command to its
arguments; in a pinch (such as for testing) you can supply an alternate font
command as an optional first argument, which will be used instead of
\verb|\musNumFont|.
In practical use it would be better and more consistent to redefine
\verb|\musNumFont| as described above.


\subsection{Figured Bass}

For figured-bass notation, use \verb|\musFig{}|.
The command takes a single argument which may contain as many space-separated
elements as you like (just like \verb|\Shortstack| from the \code{stackengine}
package, which it uses internally).

The elements in the argument will be stacked vertically and right-aligned (to
allow for accidentals preceding the numerals).
Multiple figures in a row should be enclosed in brackets if there are spaces
between them.

To align figures when there are empty slots, as in \musFig{6--5
\noFig[5--]\musSharp}, use \verb|\noFig[]|. 
It will insert horizontal space the same width as its optional argument when
typeset; the default value is the width of the numeral 5.
It does this using \verb|\hphantom|, so you should supply the same text in the
argument as the figure with which you want to align. 
The code for the example at the beginning of this paragraph was 
\verb|\musFig{6--5 \noFig[5--]\musSharp}|.

You can adjust the size and font of the figures by redefining
\verb|\musFigSize| and \verb|\musFigFont| as described above.
The command applies both commands, in that order, so if you change them, make
sure the two work together appropriately.

{\renewcommand{\arraystretch}{2.5}
\small
\begin{codetable}
    Two figures & \verb|\musFig{7 \sh5}| 
    & \musFig{7 \sh5}\\
    Two figures, accidentals after & \verb|\musFig[l]{7 \sh5}| 
    & \musFig[l]{7 5\sh}\\
    Three figures & \verb|\musFig{\fl7} 5 3}| 
    & \musFig{\fl7 5 3}\\
    Four figures & \verb|\musFig{7 5 4 2}| &
    \musFig{7 5 4 2}\\
    Voice leading & \verb|\musFig{7--6 5--4 3}| 
    & \musFig{7--6 5--4 3}\\
    Multiple figures per row 
    & \verb|\musFig{{\fl6} 5} {\noFig[\fl6] \sh}}|
    & \musFig{{\fl6 5} {\noFig[\fl6] \sh}}\\
\end{codetable}}

\subsection{Scale Degrees and Octave Numbers}

\begin{codetable}
    Scale degree & \verb|A is \musDegree{6} in C major.|
    & A is \musDegree{6} in C major. \\
    Pitch with octave & \verb|\musPitch{A}{4} is 440Hz.|
    & \musPitch{A}{4} is 440Hz. \\
\end{codetable}

For more flexible options, including traditional, non-Helmholtz notation for
pitches, use my \code{octave} package.

\section{Customization}

To create aliases for these commands, such as \verb|\quaver| for British
usage, use \verb|\NewDocumentCommand| like this:
\begin{quote}
    \verb|\NewDocumentCommand{\quaver}{}{\musEighth}|
\end{quote}
If this new LaTeX interface is unfamiliar, \verb|\newcommand| or even
\verb|\let| will still work.

\LaTeX{} programmers may wish to use the package's internal commands directly
to access more symbols from the fonts or fine-tune their appearance.
See \verb|\musSymbol|, \verb|\musAccidental|, and \verb|\musStack| in the code
listing below. 

\subsubsection{MusiX\TeX{} Symbol List}

You may use \verb|\musSymbol| to pull in any of the following symbols by
number:
\musixSymbolList

\section{Changes}

\begin{itemize}
    \item[2023-09-08] Improved customization, time signatures, figured bass,
        and documentation
        \begin{itemize}
            \item Documentation reorganized, clarified, corrected
            \item Customizable font commands used throughout:
                \verb|\musAccidentalFont|, \verb|\musNumFont|,
                \verb|\musFigSize|, and \verb|\musFigFont|
            \item \verb|\musStack| now takes an optional last argument to
                specify alignment
            \item \verb|\musMeter| now takes an optional font argument, and
                has a smaller vertical gap between numerals. It is now scaled
                to fit within the line.
            \item Spacing commands in \verb|\musSymbolMeter| are simplified
            \item \verb|\meterThreeTwo| now uses \verb|\musMeter| for the
                numbers, shifted slightly to align with the meter symbol
            \item \verb|\meterCZ| now pulls in a cursive Z symbol from the EB
                Garamond font (installed by default in TeXLive)
            \item \verb|\musFig| now includes configurable \verb|\musFigSize|
                and \verb|\musFigFont| switches, and a first optional argument
                with alignment specifier
            \item New commands for scale degress (\verb|\musDegree|) and
                pitches with Helmholtz octave (\verb|\musPitch|)
            \item For testing or extension, add command to display a list of
                symbols by number in a given font
        \end{itemize}
    \item[2020-01-29] Reduced stack gap for \verb|\musFig|
    \item[2019/11/24] Fix \verb|\MeterCThreeTwo| so that numbers are stacked
        and aligned correctly
    \item[2019/05/28] Bux fixes and new features, especially figured bass
        \begin{itemize}
            \item Fix issue 1 on Bitbucket, thanks to Christian Mondrup
                \begin{itemize}
                    \item Correct mistaken notehead symbol for values less than
                        quarter note (was hollow, now filled)
                    \item Add 32nd and 64th notes, regular and dotted varieties
                    \item \verb|\musSymbol| is now typeset inside an
                        \verb|\mbox| to fix alignment problems when used in
                        tables
                \end{itemize}
            \item New features (some requested in issue no. 1):
                \begin{itemize}
                    \item Add medieval perfect meter
                    \item Add aliases for \term{fusa} ($=$ corchea $=$ eighth note)
                    \item Replace musStack implementation with \code{stackengine}
                    \item Add full figured-bass support
                \end{itemize}
        \end{itemize}
    \item[2018/05/21] Override semantic-markup's \verb|\fl|, \verb|\sh|,
        \verb|\na| commands if it is loaded
    \item[2017/10/31] Corrected glyph used for \verb|\musHalf| and documented
        \verb|\musMeter| and \verb|\musFigures|
    \item[2017/08/29] First version on CTAN
    \item[2017/04/12] Created
\end{itemize}

\section{Code}

\VerbatimInput{musicography.sty}


\end{document}
